
%###################################### Packages and new functions definitions ###########################################%
%Encoding Package
\documentclass[11pt]{template}
\usepackage{pslatex}
\usepackage[utf8]{inputenc}
\usepackage[cyr]{aeguill}
\usepackage{xspace}
\usepackage[francais]{babel}
\usepackage[T1]{fontenc}
\usepackage{fancybox}
\usepackage{a4wide}
\usepackage[toc]{appendix}
\usepackage{CJK}

\newcommand\japonais[1]{\begin{CJK}{UTF8}{min}#1\end{CJK}}
\usepackage{color}
%Type de package de couleur : au choix. Attention n'en choisir qu'un seul à  la fois
%\usepackage{xcolor}
%\usepackage[dvipsnames]{xcolor}
\usepackage[svgnames]{xcolor}

%Packages Algoritmic
\usepackage{algorithm}
\usepackage{algorithmic}

%Packages Images & graph
\usepackage{graphicx}
\usepackage{graphics}
\usepackage{enumitem}
\usepackage{pifont}

\frenchbsetup{StandardLists=true}

\graphicspath{{./images/}}
\bibliographystyle{prsty}

\renewcommand{\algorithmicrequire} {\textbf{\textsc{\textcolor{DarkSlateGray}{In:}}}}
\renewcommand{\algorithmicensure}  {\textbf{\textsc{\textcolor{DarkSlateGray}{Out:}}}}
\renewcommand{\algorithmicwhile}   {\textcolor{DarkOrchid}{\textbf{While}}}
\renewcommand{\algorithmicdo}      {\textcolor{DarkOrchid}{\textbf{Do}}}
\renewcommand{\algorithmicendwhile}{\textcolor{DarkOrchid}{\textbf{End While}}}
\renewcommand{\algorithmicend}     {\textcolor{DarkOrchid}{\textbf{End}}}
\renewcommand{\algorithmicif}      {\textbf{\textcolor{DarkOrchid}{If}}}
\renewcommand{\algorithmicendif}   {\textcolor{DarkOrchid}{\textbf{End If}}}
\renewcommand{\algorithmicelse}    {\textcolor{DarkOrchid}{\textbf{Else}}}
\renewcommand{\algorithmicthen}    {\textcolor{DarkOrchid}{\textbf{Then}}}
\renewcommand{\algorithmicfor}     {\textbf{\textcolor{DarkOrchid}{For}}}
\renewcommand{\algorithmicforall}  {\textbf{\textcolor{DarkOrchid}{For All}}}
\renewcommand{\algorithmicdo}      {\textbf{\textcolor{DarkOrchid}{Do}}}
\renewcommand{\algorithmicendfor}  {\textbf{\textcolor{DarkOrchid}{End For}}}
\renewcommand{\algorithmicloop}    {\textbf{\textcolor{DarkOrchid}{Loop}}}
\renewcommand{\algorithmicendloop} {\textbf{\textcolor{DarkOrchid}{End Loop}}}
\renewcommand{\algorithmicrepeat}  {\textbf{\textcolor{DarkOrchid}{Repeat}}}
\renewcommand{\algorithmicuntil}   {\textbf{\textcolor{DarkOrchid}{Until}}}
\renewcommand{\algorithmicreturn}  {\textbf{\textcolor{DarkOrchid}{return}}}
\newcommand{\algorithmicelseif}{\textbf{\textcolor{DarkOrchid}{Elsif }}}
%\floatname{algorithm}{Algorithme}
\let\mylistof\listof
\renewcommand\listof[2]{\mylistof{algorithm}{List of algorithm}}



\makeatletter
\providecommand*{\toclevel@algorithm}{0}
\makeatother
\makeatletter
\def\clap#1{\hbox to 0pt{\hss #1\hss}}%
\def\ligne#1{%
  \hbox to \hsize{%
    \vbox{\centering #1}}}%
\def\haut#1#2#3#4{%
  \hbox to \hsize{%
    \rlap{\vtop{\centering #1}}%
    \rlap{\vtop{\centering #2}}%
    \clap{\vtop{\centering #3}}%
    \llap{\vtop{\centering #4}}}}%
\def\bas#1#2#3{%
  \hbox to \hsize{%
    \rlap{\vbox{\raggedright #1}}%
    \hss
    \clap{\vbox{\centering #2}}%
    \hss
    \llap{\vbox{\raggedleft #3}}}}%
\def\maketitle{%
  \thispagestyle{empty}\vbox to \vsize{%
    \haut{}{\@blurb}{}
    \vfill
    \vspace{4cm}
    \ligne{\Large \@title}
    \vspace{5mm}
    \ligne{\Large \@author}
    \vspace{1cm}
    \vfill
    \vfill
    \bas{}{\hspace{20mm} \@licence\newline}{}
  }%
  \cleardoublepage
}
	\def\date#1{\def\@date{#1}}
	\def\author#1{\def\@author{#1}}
	\def\title#1{\def\@title{#1}}
	\def\licence#1{\def\@licence{#1}}
	\def\blurb#1{\def\@blurb{#1}}
	\def\email#1{\def\@email{#1}}
	\def\logo#1{\def\includegraphics[height=3\baselineskip]{#1}}
	\date{\today}
	\author{}
	\title{}
	\licence{France}
	\blurb{}
	\email{steeve.vandecappelle@gmail.com}
	\makeatother
	\title{\begin{Huge}\textcolor{blue}{Documentation de Heimdall v2.0}\end{Huge}}
	\author{Steeve \textsc{Vandecappelle}}
	\date{3 juil 2014}
	\licence{GNU GENERAL PUBLIC LICENSE Version 3}
	\blurb{
		\includegraphics[height=3\baselineskip]{heimdall.png}
	}


%#################################################### Beginning document #################################################%
\begin{document}
\maketitle
\newpage
\strut
\newpage
\startcontents[sections]
%####################################################### Contents ########################################################%
\section*{Table des matières} 
\printcontents[sections]{l}{1}{\setcounter{tocdepth}{2}}
\newpage

%#################################################### Thanks Section #####################################################%
\section{Introduction}
Heimdall est une application de gestions de droits d'accès à divers service. La version actuelle ne s'occupe que des accès SSH.
Il fonctionne comme un bastion regroupant l'ensemble des accès aux serveurs du parc informatique. Il permet ainsi d'autoriser un utilisateur à acceder à un serveur qu'il gère (via échange de clés privées/publiques).
\\\vspace{2mm}
Un systeme de réplication de clé permet au ``bastion'' de rendre accessible le serveur en ssh.
\vspace{15mm}
Les prérequis pour permettre la réplication sont:\\
1. Un accès en SSH (via clé RSA) depuis le bastion vers le serveur cible à configurer\\
2. Un accès à un navigateur web\\
3. Un client SSH pour donner la possibilité au client de se connecter au serveur configuré.

\newpage
%################################################### Abstract Section ####################################################%
\section{Note de version}
Voici les différentes versions de Heimdall et leurs évolutions.
\vspace{8mm}\\
\textbf{v2.0}
\begin{itemize}[label=\ding{43}]
\item Passage à NodeJs
\item Nouveau mode terminal
\item Passage à redis
\item Démarrage simplifé
\item Nouvelle interface plus ergonomique
\item Optimisations
\item Internationalisation
\end{itemize}
\vspace{10mm}
\textbf{v1.0} \textbf{\textcolor{red}{La version de ce document}}
\begin{itemize}[label=\ding{43}]
\item Mode terminal désactivé au profit d'un mode web
\item Utilisation de django
\item Utilisation de celery et rabbit-mq pour gerrer une fille d'attente de traitement
\item Gestion des utilisateurs / groupes / managers de ressources
\item Les utilisateurs peuvent eux-même remonter des demandes
\item Mailer
\end{itemize}
\vspace{10mm}
\textbf{v0.1}
\begin{itemize}[label=\ding{43}]
\item Mode terminal
\item Réplication de clé
\item Gestion des utilisateurs et groupes en Base SQLLite
\end{itemize}

\newpage
\section{Installation}
\subsection{requirements}
You need to install first:
\begin{itemize}[label=\ding{43}]
\item nodejs > 0.8
\end{itemize}

\newpage
\section{Démarrage et arret}
\subsection{Configuration initiale au premier démarrage}
Avant de démarrer les services Heimdall pour la première fois il y a quelques configuration a effectuer.\\

\subsection{Démarrage et arret du serveur web}

\newpage
\section{Formation}
\subsection{Utilisateurs}
\subsubsection{Création / modification}
\subsubsection{Dépot de clés}

\subsection{Groupes}
\subsubsection{Groupes systemes et groupes de serveurs}
\subsubsection{Création / modification}

\subsection{Serveurs}
\subsubsection{Création / modification}

\subsection{Permissions}
\subsubsection{Faire une demande}
\subsubsection{Gestion des demandes}

\subsection{Administration}

\newpage
\section{Opérations de services}

\newpage
\section{Migration depuis la version antérieure}

\end{document}